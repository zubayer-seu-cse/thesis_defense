\chapter{Introduction}\label{chap:intro}
Software-Defined Networking (SDN) represents a transformative shift in network architecture by decoupling the control and data planes, enabling centralized network control, global visibility, and dynamic programmability \cite{mckeown2008openflow}. Through protocols such as OpenFlow, the SDN controller communicates with forwarding devices to install or modify flow rules that govern how packets are handled. This centralized and programmable design significantly enhances flexibility and automation in modern networks. However, it simultaneously introduces critical security vulnerabilities, as the central controller becomes a high-value target and the control channel a potential attack surface \cite{shaghaghi2016security}.

With the growing sophistication of cyber threats, SDN environments face a diverse range of intrusion attempts, including Distributed Denial of Service (DDoS), Man-in-the-Middle (MITM), probing, and brute-force attacks. These attacks exploit weaknesses in the centralized control model, potentially resulting in information theft, service disruption, eavesdropping, or even total network failure. As attack vectors evolve, conventional security solutions—such as static firewall rules or traditional intrusion detection systems—have become inadequate due to their limited adaptability and reliance on predefined signatures.

Although the use of Transport Layer Security (TLS) can protect the control channel, its computational and management overhead make it unsuitable for resource-constrained or latency-sensitive environments. Similarly, heavy cryptographic frameworks and hardware-dependent defenses often require modifications to network devices or OpenFlow itself, undermining SDN’s inherent flexibility and scalability. These limitations highlight the pressing need for lightweight, adaptive, and easily deployable mechanisms that can proactively detect and mitigate intrusions without compromising performance.

To address these challenges, this paper proposes a lightweight hybrid security framework that integrates deep learning–based anomaly detection with robust cryptographic verification. The primary defense layer employs a Temporal Convolutional Network (TCN) trained on InSDN dataset and exported in ONNX format to make it compatible to run as a controller application, to analyze network flow statistics for real-time intrusion detection. This learning-based module proactively identifies suspicious behaviors indicative of various attack patterns. Reinforcing this, a lightweight auxiliary security agent operates as a subprocess and connects with the switches and the controller, that ensures flow rule integrity using Hash-based Message Authentication Codes (HMACs) and performs periodic challenge–response verification to confirm controller authenticity.

The proposed approach is evaluated using the InSDN dataset, which contains multiple attack scenarios such as DDoS, MITM, Probe, and Brute-force. Experimental results demonstrate that the TCN achieves a detection accuracy of 99.85\%, while the HMAC-based verification introduces negligible computational overhead. These outcomes validate the proposed solution as an efficient and scalable defense mechanism, offering both proactive intrusion detection and cryptographic assurance, thereby strengthening the overall security posture of SDN environments.

\endinput