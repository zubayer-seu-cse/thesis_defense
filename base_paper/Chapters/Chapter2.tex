\chapter{Literature Review}\label{chap:literature_review}

Recent efforts to enhance the security of Software-Defined Networking (SDN) have explored both cryptographic verification and intelligent intrusion detection mechanisms. Pradeep et al. \cite{Pradeep2023EnsureS} proposed EnsureS, a flow rule verification scheme utilizing batch hashing and tag checks to achieve low-overhead rule validation. Zhou et al. \cite{Zhou2022SecureMatch} introduced SecureMatch, a cryptographic rule-matching mechanism that ensures secure rule installation with minimal latency. Similarly, Ahmed et al. \cite{Ahmed2023HMACSDN} developed a modular HMAC-based framework that verifies flow mod messages without requiring protocol or switch modifications. These works collectively emphasize the importance of preserving flow rule integrity and controller–switch trust, though they primarily focus on reactive validation rather than proactive threat detection.

In parallel, blockchain-based approaches have been adopted to provide decentralized trust and auditability in SDN. Song et al. \cite{Song2023IS2N} leveraged blockchain for securing intent-driven SDN, emphasizing decentralized control. Rahman et al. \cite{Rahman2022BCSDN} surveyed blockchain’s potential to safeguard controller communications and flow rule integrity. Poorazad et al. \cite{Poorazad2023BlockchainIDS} combined blockchain with deep learning for real-time threat detection in Industrial IoT, while Tselios et al. \cite{Tselios2025MuZeroBlockchain} applied MuZero reinforcement learning for secure controller placement with blockchain-based audit trails. Alkhamisi et al. \cite{Alkhamisi2024BCS} developed a blockchain-enabled control plane framework to detect cross-controller attacks. Although these blockchain approaches strengthen accountability and distributed trust, they often incur high latency and computational cost, limiting their applicability in lightweight or resource-constrained SDN environments.

Intrusion detection and anomaly-based techniques remain essential for identifying malicious activities within SDN. Sharma and Tyagi \cite{Sharma2023LightweightIDS} proposed a lightweight intrusion detection system for identifying MITM and DoS attacks, while Liang et al. \cite{Liang2021NIDSReview} comprehensively reviewed SDN-based IDS mechanisms against rule injection and replay threats. Benkhelifa et al. \cite{Benkhelifa2020AI4SDN} discussed the role of AI-driven security methods for preventing flow tampering and policy abuse, and Wang et al. \cite{Wang2022DeepFlowGuard} introduced DeepFlowGuard, a deep learning-based controller authentication system. These studies demonstrate the growing effectiveness of AI-based intrusion detection, though most lack an integrated approach combining anomaly detection with real-time integrity verification.

Authentication and access control have also received considerable attention. Sousa and Gonçalves \cite{Sousa2024FedAAA} proposed FedAAA-SDN, a federated identity management framework for secure cross-domain authentication, while Khan et al. \cite{Khan2021MitmDefender} developed MITM-Defender, which detects controller–switch anomalies using behavioral flow tracking. Malik et al. \cite{Malik2021SDoS} addressed DoS threats through lightweight agents deployed near switches to detect abnormal flow mod activity.

Overall, prior works have either focused on heavyweight cryptographic or blockchain frameworks that introduce scalability limitations, or machine learning-based intrusion detection systems that lack integrity assurance at the control plane. This motivates the need for a hybrid lightweight solution that seamlessly combines deep learning-based anomaly detection with HMAC-based cryptographic verification, achieving both proactive intrusion prevention and secure, low-overhead flow rule validation within SDN environments.
\endinput
