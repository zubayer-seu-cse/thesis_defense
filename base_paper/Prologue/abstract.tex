Software-Defined Networking (SDN) offers unprecedented flexibility in network management by decoupling the control and data planes, which results in centralized network control and programmability. This centralization, however, introduces significant security vulnerabilities, leading to catastrophic outcomes such as network intrusion, information theft, eavesdropping, and total network failure. As the attack method evolves relying solely on traditional firewall is insufficient. Also to defend against interception based attacks the TLS encryption is not suitable for resource constrained environments because of its computational overhead. This paper proposes a lightweight hybrid security framework to defend against network intrusion by integrating proactive deep learning-based anomaly detection with robust cryptographic verification. The primary defense layer employs a Temporal Convolutional Network (TCN), exported in ONNX format and running as a controller application, to analyze flow statistics to detect intrusion. This is reinforced by a lightweight, auxiliary agent running as a sub-process, which validates all flow rule installations using HMACs to ensure integrity and executes a periodic challenge-response protocol to validate controller authenticity, providing cryptographically-guarantee. To evaluate the proposed approach InSDN dataset is used, which consists of attack data such as DDoS, MITM, Probe and Bruteforce. The TCN in the proposed method has achieved 99.85\% accuracy with the HMAC integrity verification consuming minimal computational resource, making it appropriate as a lightweight hybrid approach to defend against network intrusion in SDN.