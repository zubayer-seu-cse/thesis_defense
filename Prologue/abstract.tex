Software-Defined Networking (SDN) has reshaped modern network management by separating the control plane from the data plane, bringing centralized programmability and fine-grained traffic control. That very centralization, opens the door to serious security risks such as, network intrusion, information theft, eavesdropping, and, in worst-case scenarios, complete network failure. As attack strategies grow more sophisticated, relying on conventional firewalls alone is no longer tenable, and while TLS encryption can protect the control channel, its computational overhead makes it a poor fit for resource-constrained SDN deployments. This thesis presents a lightweight hybrid security framework that tackles these threats on two fronts. The first layer of defense is a custom Temporal Convolutional Network (TCN) exported in ONNX format and deployed as a controller application, it inspects flow statistics in real time and flags the traffic as malicious or benign. Backing this up is an auxiliary agent that runs as a co-located subprocess, verifying every flow rule installation through HMAC-based integrity checks and periodically executing a challenge--response protocol to confirm controller authenticity. We evaluate the framework on the InSDN dataset, which covers DDoS, MITM, Probe, and Brute-force attack categories. The TCN achieves 99.85\% classification accuracy and a false alarm rate of just 0.37\%, while the HMAC verification adds negligible computational overhead, roughly 0.7~ms per operation. Taken together, these results demonstrate that the proposed TCN-HMAC approach delivers robust, multi-layered SDN security without sacrificing the real-time performance modern networks demand.