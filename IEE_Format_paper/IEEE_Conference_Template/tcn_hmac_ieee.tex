\documentclass[conference]{IEEEtran}
\IEEEoverridecommandlockouts
\usepackage{cite}
\usepackage{amsmath,amssymb,amsfonts}
\usepackage{graphicx}
\usepackage{textcomp}
\usepackage{xcolor}
\usepackage{booktabs}
\usepackage{multirow}
\usepackage[ruled,lined,linesnumbered]{algorithm2e}
\def\BibTeX{{\rm B\kern-.05em{\sc i\kern-.025em b}\kern-.08em
    T\kern-.1667em\lower.7ex\hbox{E}\kern-.125emX}}
\begin{document}

\title{TCN-HMAC: A Lightweight Deep Learning and Cryptographic Hybrid Security Framework for Software-Defined Networks}

\author{\IEEEauthorblockN{Md.\ Mahamudul Hasan Zubayer}
\IEEEauthorblockA{\textit{Department of Computer Science and Engineering} \\
\textit{Southeast University}\\
Dhaka, Bangladesh}
\and
\IEEEauthorblockN{Dr.\ Md.\ Maruf Hassan}
\IEEEauthorblockA{\textit{Department of Computer Science and Engineering} \\
\textit{Southeast University}\\
Dhaka, Bangladesh}
}

\maketitle

\begin{abstract}
Software-Defined Networking (SDN) has reshaped modern network management by separating the control plane from the data plane, bringing centralized programmability and fine-grained traffic control. That very centralization, opens the door to serious security risks such as, network intrusion, information theft, eavesdropping, and, in worst-case scenarios, complete network failure. As attack strategies grow more sophisticated, relying on conventional firewalls alone is no longer tenable, and while TLS encryption can protect the control channel, its computational overhead makes it a poor fit for resource-constrained SDN deployments. This thesis presents a lightweight hybrid security framework that tackles these threats on two fronts. The first layer of defense is a custom Temporal Convolutional Network (TCN) exported in ONNX format and deployed as a controller application, it inspects flow statistics in real time and flags the traffic as malicious or benign. Backing this up is an auxiliary agent that runs as a co-located subprocess, verifying every flow rule installation through HMAC-based integrity checks and periodically executing a challenge--response protocol to confirm controller authenticity. We evaluate the framework on the InSDN dataset, which covers DDoS, MITM, Probe, and Brute-force attack categories. The TCN achieves 99.85\% classification accuracy, a 99.97\% detection rate, and a false alarm rate of just 0.37\%, while the HMAC verification adds negligible computational overhead, roughly 0.7~ms per operation. Taken together, these results demonstrate that the proposed TCN-HMAC approach delivers robust, multi-layered SDN security without sacrificing the real-time performance modern networks demand.
\end{abstract}

\begin{IEEEkeywords}
Software-Defined Networking, Intrusion Detection System, Temporal Convolutional Network, HMAC, Network Security, Deep Learning
\end{IEEEkeywords}

\section{Introduction}

The appeal of Software-Defined Networking---programmability, centralised policy, hardware independence---comes with an uncomfortable corollary: a compromised controller can redirect, copy, or silently drop arbitrary traffic~\cite{kreutz2015sdn}. Two bodies of work have attacked this problem from opposite ends. Machine-learning IDS classify data-plane traffic but leave the OpenFlow channel wide open; TLS encrypts that channel but cannot verify individual flow-rule commands or detect application-layer intrusions~\cite{Ahmed2023HMACSDN}. Blockchain-based alternatives offer strong integrity guarantees, yet their consensus latencies, measured in seconds, are incompatible with the millisecond-scale reactions an SDN controller requires~\cite{Song2023IS2N}.

We bridge these two worlds with \textbf{TCN-HMAC}, a single lightweight system in which a Temporal Convolutional Network inspects data-plane flows while an HMAC-SHA256 mechanism authenticates every control-plane message, verifies flow rules against a shadow table, and periodically re-authenticates controller identity.

\textbf{Contributions.}
\begin{enumerate}
    \item A compact TCN architecture (156K parameters, 612\,KB) for binary SDN intrusion detection achieving 99.85\% accuracy and 99.97\% detection rate on the InSDN dataset.
    \item An HMAC-based communication integrity protocol featuring key rotation, sequence numbers, flow rule shadow-table verification, and mutual challenge--response authentication.
    \item A comparative evaluation against 15 existing IDS and SDN security approaches, demonstrating competitive detection performance and superior computational efficiency.
\end{enumerate}

\section{Related Work}

\textbf{Deep-learning IDS for SDN.}
Said et al.~\cite{Said2023CNNBiLSTM} stacked CNN with BiLSTM on InSDN, reaching 99.90\% accuracy---but the bidirectional path means the model needs future time steps, an awkward requirement for real-time detection. Kanimozhi et al.~\cite{Kanimozhi2025DRL} tried deep reinforcement learning (DDQN) on the same dataset (98.85\%) at considerably higher computational cost. Neither study addresses control-plane security.

\textbf{TCN variants for NIDS.}
Lopes et al.~\cite{Lopes2023TCN} were among the first to apply a TCN to network intrusion detection (99.75\% on CIC-IDS-2017). Follow-up work bolted on squeeze-and-excitation blocks~\cite{LiLi2025TCNSE}, attention~\cite{Benfarhat2025TCN}, bidirectional processing~\cite{Mei2024BiTCN}, and multi-head self-attention~\cite{Deng2024BiTCNMHSA}. None of these fancier variants clearly outperform a plain TCN paired with careful preprocessing.

\textbf{SDN control-plane security.}
Ahmed et al.~\cite{Ahmed2023HMACSDN} proposed HMAC-SHA256 for OpenFlow message authentication but provided no IDS layer. Blockchain-backed frameworks~\cite{Song2023IS2N, Rahman2022BCSDN} deliver strong guarantees at the price of latency and resource overhead that rule out real-time operation.

To our knowledge, TCN-HMAC is the first system to fold temporal-convolutional intrusion detection and cryptographic control-plane protection into a single lightweight package.

\section{Proposed Framework}

\subsection{Architecture Overview}

TCN-HMAC operates at two security boundaries within the SDN architecture (Fig.~\ref{fig:arch}):

\begin{itemize}
    \item \textbf{Data $\to$ Control plane:} the TCN-IDS inspects traffic forwarded via \texttt{Packet\_In} messages, classifying each flow as benign or malicious.
    \item \textbf{Control $\leftrightarrow$ Data plane:} the HMAC mechanism authenticates every \texttt{Flow\_Mod} and \texttt{Stats\_Reply} message between controller and switches.
\end{itemize}

An Auxiliary Agent runs as a sidecar process alongside the controller, handling HMAC key management, flow rule verification against a shadow table, and periodic challenge--response authentication.

\begin{figure}[t]
    \centering
    \includegraphics[width=\columnwidth]{../../Figures/system_architecture.drawio.png}
    \caption{TCN-HMAC framework architecture.}
    \label{fig:arch}
\end{figure}

\subsection{TCN Model}

The network comprises six dilated causal residual blocks (dilation rates $d \in \{1,2,4,8,16,32\}$), each containing two Conv1D layers (64 filters, kernel 3) with batch normalization, ReLU, and SpatialDropout1D(0.2), connected through a residual skip path. Global average pooling feeds into two dense layers (128 and 64 units with dropout 0.2) followed by a single sigmoid output. The full model has 156,737 parameters and occupies 612\,KB.

Input consists of 24 principal components derived by applying PCA (retaining 95.43\% variance) to 48 cleaned features from the InSDN dataset's original 84. The preprocessing pipeline includes infinite-value replacement, zero-variance and near-constant feature removal, Pearson correlation thresholding ($|r|>0.95$), StandardScaler normalization, and inverse-frequency class weighting (benign: 1.4258, attack: 0.7700).

\subsection{HMAC Communication Integrity}

\textbf{Key Establishment.} Upon switch connection, the Auxiliary Agent generates a per-switch symmetric key $K_{sw}$, encrypts it with the controller's RSA public key along with an HMAC tag over the pre-shared agent--controller key, and transmits the bundle securely.

\textbf{Message Authentication.} Every OpenFlow message carries a tag:
\begin{equation}
    \text{tag} = \text{HMAC-SHA256}(K, M \| \text{seq} \| \text{ts})
\end{equation}
where $K$ is the directional session key, $\text{seq}$ enforces ordering, and $\text{ts}$ provides freshness. Constant-time comparison guards against timing side-channels.

\textbf{Flow Rule Verification.} The controller keeps a shadow copy of each switch's flow table. Periodic \texttt{Flow\_Stats\_Request} queries detect unauthorized additions, modifications, or deletions; discrepancies trigger automatic re-installation and alerts. The agent independently verifies each flow rule cookie:
\begin{equation}
    C_{\text{exp}} = \text{HMAC}_{K_{sw}}(\textit{eth\_src} \| \textit{eth\_dst} \| \textit{out\_port})
\end{equation}

\textbf{Challenge--Response.} Switches periodically send a nonce $c$ to the controller; the controller returns $\text{HMAC}_{K}(c \| \text{id})$. Verification failure triggers failover to a backup controller.

\textbf{Key Rotation.} Forward-secure rotation derives $K^{(t+1)} = \text{HMAC}(K^{(t)}, r_t)$ every $T_{\text{rot}}$ seconds, ensuring compromise of a current key does not expose past communications.

\section{Experimental Setup}

\textbf{Dataset.} InSDN~\cite{elsayed2020insdn} contains 343,889 labeled samples (84 features) spanning benign traffic and four attack categories: DDoS, MITM, Probe, and Brute-force. After a 15-stage preprocessing pipeline (deduplication, cleaning, scaling, PCA), 182,831 samples with 24 features remain. These are split 70/10/20 for training, validation, and testing (36,567 test samples).

\textbf{Training.} Implementation uses TensorFlow 2.19.0 on Google Colab (NVIDIA T4 GPU). The Adam optimizer ($\eta = 10^{-3}$), binary cross-entropy loss with class weights, batch size 2,048, early stopping on validation AUC (patience 15), and ReduceLROnPlateau scheduling are employed. Training converges in roughly 30 epochs (approximately 5 minutes).

\section{Results}

\subsection{Classification Performance}

Table~\ref{tab:results} presents the held-out test-set metrics.

\begin{table}[t]
\centering
\caption{TCN\_InSDN Test-Set Performance}
\begin{tabular}{@{}lr@{}}
\toprule
\textbf{Metric} & \textbf{Value} \\
\midrule
Accuracy        & 99.85\% \\
Precision       & 99.80\% \\
Recall (DR)     & 99.97\% \\
Specificity     & 99.63\% \\
F1-Score        & 99.89\% \\
AUC-ROC         & 0.9999  \\
FAR             & 0.37\%  \\
MCC             & 0.9966  \\
\bottomrule
\end{tabular}
\label{tab:results}
\end{table}

The confusion matrix (Fig.~\ref{fig:cm}) records 12,776 true negatives, 23,737 true positives, 47 false positives, and only 7 false negatives out of 36,567 samples. The model misses just 3 attacks in every 10,000.

\begin{figure}[t]
    \centering
    \includegraphics[width=0.65\columnwidth]{../../Figures/results/plot_confusion_matrix.png}
    \caption{Confusion matrix on the held-out test set (36,567 samples).}
    \label{fig:cm}
\end{figure}

\subsection{Training Dynamics}

Validation accuracy exceeds 99.5\% within 5 epochs (Fig.~\ref{fig:curves}). Training and validation curves closely overlap throughout, confirming that generalization holds without overfitting. AUC-ROC stabilizes above 0.999 by epoch 10.

\begin{figure}[t]
    \centering
    \includegraphics[width=0.75\columnwidth]{../../Figures/results/plot_accuracy.png}
    \caption{Training and validation accuracy over 30 epochs.}
    \label{fig:curves}
\end{figure}

\subsection{HMAC and System Overhead}

HMAC-SHA256 computation requires roughly 2\,$\mu$s per message, sustaining over 500,000 messages per second on a single core---well in excess of typical OpenFlow rates (1K--10K\,msg/s). The Auxiliary Agent adds 0.7\,ms to control latency, 4.4\% additional CPU usage, and 13.7\,MB RAM, while successfully detecting all injected invalid flow rules during testing. End-to-end per-flow latency totals approximately 170\,$\mu$s (Table~\ref{tab:latency}).

\begin{table}[t]
\centering
\caption{End-to-End Per-Flow Latency Breakdown}
\begin{tabular}{@{}lr@{}}
\toprule
\textbf{Component} & \textbf{Latency} \\
\midrule
HMAC Verification  & $\sim$2\,$\mu$s \\
Feature Extraction  & $\sim$50\,$\mu$s \\
Preprocessing       & $\sim$10\,$\mu$s \\
TCN Inference (GPU) & $\sim$100\,$\mu$s \\
Decision + Response & $\sim$5\,$\mu$s \\
HMAC Signing        & $\sim$2\,$\mu$s \\
\midrule
\textbf{Total}      & $\sim$\textbf{170\,$\mu$s} \\
\bottomrule
\end{tabular}
\label{tab:latency}
\end{table}

\section{Comparative Analysis}

Table~\ref{tab:comparison} benchmarks TCN-HMAC against 14 existing approaches. On InSDN, TCN-HMAC posts the highest detection rate (99.97\%) among all models. Said et al.~\cite{Said2023CNNBiLSTM} report 0.05\% higher accuracy, but their CNN-BiLSTM requires approximately 2\,M parameters (12$\times$ ours) and roughly 2\,ms inference (12$\times$ slower), and offers no control-plane protection.

\begin{table}[t]
\centering
\caption{Performance Comparison with Existing Approaches}
\resizebox{\columnwidth}{!}{%
\begin{tabular}{@{}llrrrr@{}}
\toprule
\textbf{Model} & \textbf{Dataset} & \textbf{Acc.} & \textbf{Rec.} & \textbf{F1} & \textbf{Auth.} \\
\midrule
\textbf{TCN-HMAC (ours)} & \textbf{InSDN} & \textbf{99.85} & \textbf{99.97} & \textbf{99.89} & \textbf{Yes} \\
CNN-BiLSTM~\cite{Said2023CNNBiLSTM} & InSDN & 99.90 & 99.90 & 99.90 & No \\
DNN Ensemble~\cite{Ataa2024SDNDL} & InSDN & 99.70 & 99.70 & 99.67 & No \\
TCN+Att~\cite{Benfarhat2025TCN} & CIC-IDS & 99.73 & 99.69 & 99.70 & No \\
BiTCN-MHSA~\cite{Deng2024BiTCNMHSA} & CIC-IDS & 99.72 & 99.72 & 99.70 & No \\
CNN-LSTM~\cite{Shihab2025CNNLSTM} & CIC-IDS & 99.67 & 99.67 & 99.67 & No \\
TCN-SE~\cite{LiLi2025TCNSE} & NSL-KDD & 99.62 & 99.62 & 99.60 & No \\
BiTCN~\cite{Mei2024BiTCN} & CIC-IDS & 99.60 & 99.60 & 99.57 & No \\
Hybrid DL~\cite{Kumar2025HybridDL} & CIC-IDS & 99.45 & 99.45 & 99.43 & No \\
CNN-GRU~\cite{Yang2024CNNGRU} & NSL-KDD & 99.35 & 99.35 & 99.27 & No \\
LSTM~\cite{Basfar2025LSTM} & NSL-KDD & 99.23 & 99.23 & 99.11 & No \\
CNN~\cite{Ahmad2021CNN_SDN} & InSDN & 99.20 & 99.20 & 99.15 & No \\
DRL~\cite{Kanimozhi2025DRL} & InSDN & 98.85 & 98.85 & 98.87 & No \\
DT/RF~\cite{elsayed2020insdn} & InSDN & 98.50 & 98.50 & 98.45 & No \\
\bottomrule
\end{tabular}%
}
\label{tab:comparison}
\end{table}

Among the TCN family specifically, our model outperforms TCN-SE (99.62\%), TCN+Attention (99.73\%), BiTCN (99.60\%), and BiTCN-MHSA (99.72\%)---all without attention layers or bidirectional wiring. The lesson seems to be that a disciplined preprocessing pipeline (StandardScaler, PCA, class weighting) paired with a clean architecture can match or beat structurally fancier alternatives.

At 612\,KB the model is 5--30$\times$ smaller than comparable deep-learning IDS models, and at 0.17\,ms its inference is the fastest in the comparison, supporting over 5,000 flows per second on a single GPU. Crucially, TCN-HMAC remains the only evaluated approach that also protects the control channel.

\section{Conclusion}

We presented TCN-HMAC, a hybrid security framework that pairs a compact TCN-based IDS (99.85\% accuracy, 99.97\% detection rate, 612\,KB, 0.17\,ms inference) with HMAC-SHA256 control-plane authentication at roughly 2\,$\mu$s per message. Against 15 published approaches, it delivers competitive---often the best---detection accuracy at the lowest computational cost, and it is the only system that also secures the controller--switch channel.

Natural next steps include cross-dataset validation (NSL-KDD, CIC-IDS-2017, UNSW-NB15), multi-class attack-type identification, adversarial robustness evaluation, and production-grade integration with open-source controllers such as ONOS and Ryu.

\bibliographystyle{IEEEtran}
\bibliography{references}

\end{document}

