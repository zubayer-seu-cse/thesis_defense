%MARGIN FOR ALGORITHM
\IncMargin{1em}
%START OF FUNCTION
\begin{func}
\begin{algorithm}[!p]
%%%%SETTING KEYWORDS FOR STYLING THEM IN THE FUNCTION%%%%
%%%%%%%%%
%\SetKwData{x} {y} sets a data/datastructure y to be refered as x in the algorithm. See where subbst has been used in sourcecode and SliceUpperBoundBST in the algorithm.
%%%%%%%%%

\SetKwData{subbst} {SliceUpperBoundBST}
%%%%%%%%%
%\SetKwFunction{x} {y} sets a function y to be refered as x in the algorithm. See where scanslab has been used in sourcecode and ScanSlab in the algorithm.
%%%%%%%%%

\SetKwFunction{scanslab} {ScanSlab}

%%%%%%%%%
%Usage of Input,Output, Data( Persistent data like database)
%%%%%%
\KwInput{ A set of slices $S_{slice}$ }
\KwOutput{Whatever}
\KwData{$xyz$}
%%%%LINEGAPE
\BlankLine
%%%%FOR LOOP BLOCK
\For{ \upshape{each} $s_i$ \upshape{in} $S_{slice}$}
{
    %%%% \gets means assignment
    $s_i.R \gets \text{ the set of rectangles currently intersecting with } s.i$\;
    $(s_i.S_{slabs}, g_{maxub}) \gets $ \scanslab{$s_i.R$}\;
    $\subbst.update(s_i.id,g_{maxub})$\;
    $s_i.p_{c} \gets null$\;
    $s_i.lazy \gets false$\;
    $s_i.maxregsearched \gets false $\;
}
%%%%%CAPTIONING THE FUNCTION
\caption{PrepareSlices$(S_{slice})$}
\label{func:prepslice}
%END OF FUNCTION
\end{algorithm}
\end{func}
%MARGIN FOR ALGORITHM
\DecMargin{1em}